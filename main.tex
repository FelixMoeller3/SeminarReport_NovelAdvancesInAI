\documentclass{article}
\usepackage[utf8]{inputenc}
\usepackage[backend=biber,
style=numeric,
citestyle=numeric]{biblatex}
\usepackage{url}
\usepackage{csquotes}
\usepackage{pgfplots}
\usepackage{pgfplotstable}
\pgfplotsset{compat=1.7}
\usepackage{tikz}

\addbibresource{citations.bib}


\title{Detecting Spurious Correlations in Image Annotations}
\author{Felix Möller}
\date{August 10th, 2021}

\begin{document}

\maketitle
\tableofcontents
\newpage
\section{Abstract}
Neural networks and especially Convolutional Neural Networks (CNN) have revolutionized the field of computer vision in the recent years. While neural networks become more refined every year they also become more complex and difficult to interpret. However, interpretability is important because it allows researchers to investigate whether their models make their predictions by "learning from the right thing" or not. Surprisingly, recent research has shown that many models' predictions rely on so-called spurious correlations. In this report, I will present overview over the existing methods and compare them with regard to their usability. Next, I will show what should and what should not be done to mitigate the impact of spurious correlations on classification accuracy and finally I will propose a workflow to handle datasets that could contain spurious correlations. 
\section{Introduction}
\subsection{What is a spurious correlation (s.c.)?}
Before we have a look at the impact of spurious correlations on image datasets we first need to focus on what spurious correlations are. According to \cite{sc_def} a spurious correlation is a \enquote{connection between to variables that appears causal but is not}. This means that there \textit{appears} to be a logical explanation for the co-movement of both variables when the correlation is in fact completely random. The section \enquote{Why is detecting s.c.s in image datasets difficult?} mention why this is especially problematic in a machine learning context.
In this report, I will only have a look at spurious correlations that exist between a given feature of the dataset (e.g. the background of the image)and the target value (e.g. the age of a person) as these are the only spurious correlations that might cause the classifier to make wrong generalizations.

\subsection{Common types of s.c.s in image datasets}
Modern research has worked out two commonly occurring types of spurious correlations. The first one is  between \textbf{image angle} and \textbf{target} \cite{5995347}. In datasets that contain such a spurious correlation, the target is often shown from a similar angle (e.g. in an image dataset made up of cars all cars could be shown from the front) or in the same part of the image (e.g. most of the time the object of interest is in the centre of the image). Spurious correlations between image angle and target can be problematic because an image recognition model trained on a dataset which contains such a spurious correlation might struggle to classify the target object when it is shown from an angle that was not present in the training dataset. \\
The next commonly occurring spurious correlation between \textbf{context} and \textbf{target} \cite{Singh_2020_CVPR}. Spurious correlations between context and target occur in datasets where a non-target object co-occurs with the target object. One example would be cars which are commonly depicted with people. A classifier trained on a dataset with a spurious correlation between context and target might infer that the target value has to occur with its spuriously correlated non-target object. So in our car example, a classifier might only detect a car if a person is also present in the image.

\subsection{Why study spurious correlations?}
Datasets containing spurious correlations can significantly diminish the accuracy of a model compared to training on a dataset where no spurious correlations are present. Kim et al. have shown how much varying degrees of the same spurious correlations impact the classification accuracy of a model \cite{Kim_2019_CVPR}. They planted a spurious correlation between a digit and its color into the classic MNIST dataset \cite{mnist}, meaning they assigned a mean color to every digit in the training set and then drew from $N(mu_{digit},{\sigma^2}_{train})$ where $mu_{digit}$ is the mean color of the given digit and ${\sigma^2}_{train})$ is the variance of the normal distribution which was altered between iterations. The digits in the test dataset were assigned a random color (i.e. the mean color was not digit-specific), indicating that there is no spurious correlation in the test dataset. The degree of the spurious correlation between digit and color in the training set depends on ${\sigma^2}_{train}$ with a low value indicating a high degree of the spurious correlation and vice-versa. The evaluation results can be seen in Figure \ref{fig:mnist_graph}. Although the spurious correlation would most likely be noticed in practice for very low $\sigma^2$, it is clear to see that even for the highest $\sigma^2$

\begin{figure}[!ht]
    \centering
    \begin{tikzpicture}
    \begin{axis}[
        xlabel = $\sigma^2$,
        ylabel = Accuracy,
        width=10cm,
        height=7cm,
    ]
    \addplot[color=red, mark=.] coordinates {
        (0.02,0.4)
        (0.025,0.48)
        (0.03,0.6)
        (0.035,0.66)
        (0.04,0.73)
        (0.045,0.8)
        (0.05, 0.84)
    };
    \end{axis}
    \end{tikzpicture}
    \caption{Effect of a spurious correlation between color and digit in the classic MNIST dataset on classification accuracy. A lower variance indicates a more severe spurious correlation and vice-versa.}
    \label{fig:mnist_graph}
\end{figure}

\subsection{Why is detecting s.c.s in image datasets difficult?}

\section{Detecting spurious correlations}
\subsection{Overview}
\subsection{Crowdsourcing}
\subsection{Grad-CAM}
\subsection{TCAV}
\subsection{Comparing existing methods to detect spurious correlations}

\section{Mitigating the impact of spurious correlations}
\subsection{Why increasing model size exacerbates spurious correlations}
\subsection{How to mitigate the impact of spurious correlations between context and target}

\section{How to deal with datasets that might contain spurious correlations}

\section{Conclusion}

\printbibliography

\end{document}
